%----------------------------------------------------------------------------------------
%   OpenCOR
%----------------------------------------------------------------------------------------

\providecommand{\rootfolder}{../..} % Relative path to main.tex
\documentclass[\rootfolder/main.tex]{subfiles}
\begin{document}
\section{SED-ML}

The SED-ML language allows the researcher to specify all stages of the simulation itself, independent of the model that is simulated.
An experiment described with SED-ML consists of several components.

\begin{enumerate}
    \item The Model class defines which models should be used in the simulation.
    \item The Change class allows models to be tweaked for a particular simulation, including changes to attributes or additions to the model specification itself.
    \item The Simulation class specifies which of a number of predefined simulation algorithms to use for the experiment. \cite{sedml-specification}
\end{enumerate}

The simulation process broadly consists of three steps:

\begin{enumerate}
    \item Select the model to use. Any required fine-tuning, such as changing the models initial values, can be applied without requiring a change to the model specification itself.
    \item Specify the simulation algorithm that should be used for the experiment.
    \item Apply post-processing
        Here, the raw output from the simulation steps can be processed. For example, the data can be normalised,
        correlations can be calculated and so on, before the final plots or reports are generated.
\end{enumerate}

The project behind OpenCOR, called the Physiome Project, expends significant effort to reproduce the models used in scientific publications.
SED-ML is used to specify the exact simulation parameters needed to reproduce a result the way it appears in literature.
Indeed, models in the Physiome library are curated and evaluated by how closely they mimick the results they intend to reproduce.

\end{document}
