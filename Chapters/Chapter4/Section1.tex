%----------------------------------------------------------------------------------------
%   OpenCOR
%----------------------------------------------------------------------------------------

\providecommand{\rootfolder}{../..} % Relative path to main.tex
\documentclass[\rootfolder/main.tex]{subfiles}
\begin{document}
\section{OpenCOR}

OpenCOR is an open source simulation package made for the biology field.
It is modular to the point where all features are in the form of plugins \cite{10.3389/fphys.2015.00026}.
Any description of OpenCOR would be incomplete without a description of the language it uses to describe models, and the language used for describing simulation setups.
OpenCOR uses CellML and SED-ML, both XML-based markup languages, to describe the model and the simulation setup respectively.

\subsection{CellML}


\subsubsection{Introduction}

CellML \cite{cuellar2003} is an XML-based format that is used to describe the structure of models.
The information included in the model can broadly be described as twofold.
First, it includes the various parts of the model, and how these are combined into the complete model.
This information is specified separately from the simulation specification, as we shall see later.
Second, the language allows for metadata to be added to the model, in order to categorize and organize models and model parts.

Being an XML-based standard, CellML is extensible.
In addition to the above, CellML allows for model designers to add information written in other XML-based languages to the model.
This can be equations written in MathML, textual and graphical documentation in HTML (currently planned), data sets and so on.

While CellML is designed to be domain independent, the language is flexible enough that is is used to implement purely biological as well as electrochemical and mechanical models \cite{cuellar2003}

\subsubsection{Model structure}

CellML models are constructed as a network of interconnected components.
The components may contain variables as well as mathematical equations specified with MathML.
Variables belong to a single component, but may be mapped to a variable in another component by specifying a connection.
Two components may only have one connection, and all variable mappings must therefore be declared in the same place
This enforces a level of organization to the sometimes complex mappings between components \cite{cuellar2003}.

Variables in CellML are always expressed in units.
This ensures rigor when shared models are connected, as incompatible variables cannot be mapped to one another.
For example a variable declared with length units, say meters, cannot be mapped to another variable with mass units, for instance kilograms.
However, it is perfectly possible to connect a variable using kilograms to one using pounds, and have the units automatically converted.
This is possible by defining new units in terms of the dictionary of standard units included with CellML, which include the base and derived SI units.
This allows researchers to work in their preferred units, and still share their work with the rest of the scientific community \cite{cuellar2003}.

Models can be structured using grouping.
CellML provides two types of groups, encapsulation and containment.
In an encapsulating group, only one component in the group is accessible to components outside the group, in effect acting as an interface.
By using a containment group the modeller can specify that one component contains another, which allows a physical organization of the model.

\subsubsection{Metadata}

Metadata in CellML models can include information about the author, references to literature and various information that puts the model into the proper biological context \cite{cuellar2003}.
This information allows other researchers to search for

\subsection{SED-ML}

The SED-ML language allows the researcher to specify all stages of the simulation itself, independent of the model that is simulated.
An experiment described with SED-ML consists of several components.

\begin{enumerate}
    \item The Model class defines which models should be used in the simulation.
    \item The Change class allows models to be tweaked for a particular simulation, including changes to attributes or additions to the model specification itself.
    \item The Simulation class specifies which of a number of predefined simulation algorithms to use for the experiment. \cite{sedml-specification}
\end{enumerate}

The simulation process broadly consists of three steps:

\begin{enumerate}
    \item Select the model to use. Any required fine-tuning, such as changing the models initial values, can be applied without requiring a change to the model specification itself.
    \item Specify the simulation algorithm that should be used for the experiment.
    \item Apply post-processing
        Here, the raw output from the simulation steps can be processed. For example, the data can be normalised,
        correlations can be calculated and so on, before the final plots or reports are generated.
\end{enumerate}


\end{document}
