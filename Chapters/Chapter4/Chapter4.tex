% Chapter Template
\providecommand{\rootfolder}{../..} % Relative path to main.tex
\documentclass[\rootfolder/main.tex]{subfiles}
\begin{document}
\chapter{Matlab/Simulink} % Main chapter title

\label{ch:matlab} % Change X to a consecutive number; for referencing this chapter elsewhere, use \cref{ChapterX}

\section{Language features}

Matlab offers several different programming paradigms, and both numerical and symbolic mathematics.
This chapter will review both of these approaches.

In addition to the Matlab programming language, Matlab also allows for modeling using graphical blocks in Simulink.
However, it is noteworthy that these are not one and the same.
While \cref{ch:modelica} showed that in Modelica, graphical models are a representation of a text based specification, this is not the case in Matlab and graphical Simulink models are stored in a binary format.
It is possible to create models using state-space representations, transfer functions, algorithms or a combination thereof in the Matlab language.
It is also possible to create models using graphical components, either causal or acausal, as will be shown.
Furthermore, models created in the Matlab language can be included as components in a graphical model, using what is called an s-function.

To illustrate some features of the language, the example with the coupled drive-shaft from \cref{ch:modelica} will be revisited.
The equations for this system were shown in \cref{eq:inertial}.
To implement this system in Matlab, it is first changed into state-space form.
Let $\vec{x}{t}$ be the states, $\vec{u}{t}$ be the inputs and $\vec{y}{t}$ be the outputs of the system; then, any such system can be represented in the following form:

\loadeq{\rootfolder/Equations.tex}{003} % State-space representation of the system in vector/matrix form

The equations shown in \cref{eq:inertial} will be transformed into state-space form.
\Cref{lst:matlab-inertial} shows \cref{eq:inertial} implemented in the Matlab language, with the same example values used in \cref{fig:modelica-inertial}.

\begin{listing}[ht]
    \inputminted{matlab}{\rootfolder/Models/Matlab/InertialMatlab.m}
    \caption{Matlab implementation of a simple drive-shaft system.\label{lst:matlab-inertial}}
\end{listing}

This creates a state-space object which can be used with a number of other methods available in Matlab.
It is also possible to transform this into a transfer function object, and use any number of Matlab methods made for analyzing these.
The system can be analyzed to find initial response, step response and so forth.
Additionally, Matlab includes toolboxes with a rich set of tools for control system creation, which can analyze the system and help the system designer do controller tuning.

\subsection{Discontinuities}

To illustrate handling of discontinuous systems in Matlab, the example with the bouncing ball, shown in \cref{lst:modelica-discontinuous}, will be revisited and implemented in Matlab code.
Matlab allows for simulation of discontinuous ODEs by allowing the user to specify points where the simulation should stop and reinitialize.
In concept, this is the same as shown previously, but Matlab requires that the user supplies an event function.
The simulation is interrupted when the output from the event function crosses zero, which in this example occurs when the ball impacts the staircase.
This interrupts the \emph{ode45()} function call, and makes it possible to set new initial values.

\begin{listing}[ht]
    \inputminted{matlab}{\rootfolder/Models/Matlab/BouncingBall.m}
    \caption{Matlab implementation of the bouncing ball model, illustrating discontinuous behavior.\label{lst:matlab-discontinuous}}
\end{listing}

As can be seen, Matlab couples the discontinuous behavior to the simulation of the model, as opposed to allowing it to be specified as part of the model itself.
It is necessary to interrupt the solver, set new initial values, and then run the solver again to calculate the next bounce of the ball, in effect requiring the user to micromanage the simulation.
This is somewhat more involved than the approach seen in \cref{ch:modelica}.
Note that it is only possible to supply a single event function to the solver.
While this is fine for a simple example such as this, it could be limiting in a more complicated scenario.

\section{Simulink modeling}

In addition to specifying the state-space or transfer function representations of the system in the Matlab language, it is possible to specify
the system equations using block components in the graphical interface Simulink.
The components used in Simulink are purely causal.
Because of this, the system must be built from components such as adders, multipliers and integrators.
To compare this with the acausal modeling paradigm shown earlier, the same system is also implemented in Simulink.

Generally, the simplest approach when implementing a system of differential equations in Simulink is to start with the integration blocks, and work backwards.
Unfortunately, as the model must be made from low-level components, the possibility of making a mistake is that much greater.
In the scope of this project significantly more time was required to create this model compared to the acausal implementations shown elsewhere in this report.

\begin{figure}[ht]
    \iimage{Figures/InertialSimulink}
    \caption{Simulink implementation of a simple drive-shaft system.\label{fig:simulink-inertial}}
\end{figure}

The block diagram equivalent of \cref{eq:inertial} is shown in \cref{fig:simulink-inertial}.
Note that the input torque is added to the term for $J_{1} \der{\omega_{1}}$.
While adding an input term in Modelica was as simple as connecting the input source to the left-hand side of the first inertia component,
adding an input term in this model requires an understanding of the underlying equations and correctly identifying the proper way to change the model.

The model is expanded in the same way as was done previously, in \cref{ch:modelica}, by adding a reference source and a controller to drive the output state of the system to the desired value.
The plant has been turned into a subsystem to reduce clutter in the model.
Note that the plant outputs $\omega$ directly while previously a sensor was added, which differentiated $\varphi$ to solve for $\omega$.
As Matlab solves this numerically, adding a differentiator is troublesome and was skipped in the example.

\begin{wrapfigure}{R}{0.5\columnwidth}
    \iimage[0.48]{Figures/InertialSimulinkExpanded}
    \caption{Simulink implementation of the drive-shaft system, expanded with a reference signal and controller.\label{fig:simulink-inertial-expanded}}
\end{wrapfigure}

\Cref{ch:modelica} showed how partial models could be instantiated with specific components.
This is useful, for example, to compare variants of two controllers.
In order to replace parts of models in Simulink there are a couple of approaches that can be used.
First, for simple components, it is possible to use the \emph{replace\_block()} function.
\Cref{lst:matlab-replace} shows how to use this to replace the step function in \cref{fig:simulink-inertial-expanded} with a different reference signal.

\begin{listing}[ht]
    \inputminted{matlab}{\rootfolder/Models/Matlab/ReplaceSource.m}
    \caption{Matlab code showing how to replace a module programmatically.\label{lst:matlab-replace}}
\end{listing}

This works for simple changes, but is limited to single blocks.
To replace subcomponents, it is necessary to introduce a variant subsystem.
This is a special type of subsystem that contains one or more plug-compatible variants.
Simulink connects one of the variants to the inputs and outputs of the subsystem based on the value of a variable.
This allows for programmatic switching between the variants.
The selection of a variant can also be made conditional on the choice of another component variant, or any other logical expression for that matter.
For example, it is possible to configure the system to use a Kalman filter only when a certain controller variant is active.
Note that the example also shows an empty controller variant, which makes it possible to run the system with no controller active.

\begin{figure}[ht]
    \iimage{Figures/InertialSimulinkBlowup}
    \caption{Simulink implementation of the drive-shaft system with variant controller subsystem.\label{fig:simscape-inertial-blowup}}
\end{figure}

This can be combined with running simulations of Simulink models from Matlab scripts in order to compare the effect of various signal sources on the behavior of the model.

Models in Matlab/Simulink can be made modular, and components can have multiple implementations.
There is, however no mechanism for the types of partial model implementation shown in \cref{fig:modelica-inertial-architecture}.
In effect, this means that while it is possible to create highly configurable models, any such changes requires that the original model is edited.
Furthermore, the Simulink file is in a binary format.
Therefore, it is not possible to do these changes with a simple text replacement, as could have done with the model from \cref{lst:modelica-inertial-instantiation} in \cref{ch:modelica}.

\section{Simscape modeling}

In addition to the causal modeling done in Simulink above, Matlab also comes with an expansion called Simscape.
In Simscape, it is possible to model using acausal components like those shown in \cref{ch:modelica}.
\Cref{fig:simscape-inertial} shows what the inertial model looks like in Simscape.

\begin{figure}[ht]
    \iimage{Figures/InertialSimscape}
    \caption{Simscape implementation of a simple drive-shaft system.\label{fig:simscape-inertial}}
\end{figure}

Simscape components are not signal-compatible with those in Simulink, but using special converter components it is possible to interface Simscape models with Simulink models.
This way, it would for example be possible to connect the acausal plant implementation above to a causal controller implementation.
As acausal modeling has already been described in \cref{ch:modelica}, Simscape modeling will not be covered in further detail.

\section{Simulation}

When simulating a model in Matlab the procedure is somewhat different depending on whether the model is defined in Matlab code, in Simulink or in Simscape.

\subsection{Simulink}

When simulating a Simulink model, the compiler first determines the values of any variables used in the model.
In the example, several values are set in a Matlab script and the compiler must populate the model with the numerical values before proceeding.
In the case of symbolic expressions, these must be resolved to determine the resulting numerical value that is used in the simulation.
The data type, numeric type and dimensions of signals are also checked, to ensure that every block in the model is compatible with the signals that are applied to it.

\begin{figure}[ht]
    \iimage{Figures/InertialSimulinkOrder}
    \caption{Illustration showing the Simulink execution order for the plant subsystem from the drive-shaft system.\label{fig:simulink-inertial-order}}
\end{figure}

Consequently, the algorithm determines the sorted execution order for the model, so that blocks are executed in an order that reflects their data dependencies.
This turns the graphical model into a sequentially executed program that can be simulated by a computer.
\Cref{fig:simulink-inertial-order} shows the execution order for the model.
The execution order is determined by the algorithm, but the designer can influence it by assigning higher priority to a block.
It is also possible to designate a subsystem as atomic, so that the whole subsystem is executed as a single unit.
Beyond this, virtual blocks, such as subsystems, are flattened and do not influence the execution order.
Finally, the sample times are determined for blocks that have inherited sample times.

After the model is compiled, the Simulink engine allocates working memory for the model execution, as well as memory for run-time data that is to be stored.
The method execution list, which lists the most efficient execution order of the model blocks, is generated from the sorted order lists found previously.

After model linking, the algorithm proceeds to the simulation loop phase.
The loop phase is separated into two sub-phases: the initialization phase and the iteration phase.
First, during initialization, the system determines the initial states for the system and the output values that need to be computed.
Then, during the loop iteration phase, the following takes place:

\begin{enumerate}
    \item Computation of outputs.
    \item Computation of the model states.
        This step is performed by invoking a solver.
        Which solver is chosen depends on whether the model has no states, only continuous states, only discrete states or both types of states.
        The user can specify which continuous and discrete solver to use, or the system can be asked to determine the optimal solver automatically.
        The solvers used by Simulink are the same as those used when implementing an ODE in Matlab code.
    \item Discontinuity checks.
    \item Computation of the time until the next time step.
\end{enumerate}

These steps are repeated until the simulation end-time is reached.

\subsection{Simscape}

Simulation of Simscape models is somewhat more involved, as the model is acausal.
Some model validation is performed for causal models, as shown above, but for acausal models the procedure is more involved.
When simulating a Simscape model, the system ensures that all blocks are connected.
It is not allowed to leave conserving ports, such as those representing hydraulic flow or energy flow, unconnected.
Models containing fluid or gas flow networks require that no more than one block, which specifies the properties of the fluid or gas, is connected to each network.

After this, the system creates the physical network.
This is based on two types of properties.

\begin{itemize}
    \item Through-properties (e.g. current) are those that are measured in series with an element.
    \item Across-properties (e.g. voltage) are those that are measured in parallel with an element.
\end{itemize}

The system follows the rule that for every through-variable, the sum of the values going into a node equal those going out.
Conversely, for every across-variable, two directly connected elements have the same value.
The system creates the physical network following these two principles.

The system equations are then developed from this network, and the equations are simulated in a similar fashion to the procedure explained for normal Simulink models.
Overall, the process is similar to that described in \cref{ch:modelica}.

\end{document}
