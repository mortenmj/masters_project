% Chapter Template
\providecommand{\rootfolder}{../..} % Relative path to main.tex
\documentclass[\rootfolder/main.tex]{subfiles}
\begin{document}
\chapter{Conclusion} % Main chapter title

\label{Chapter06} % Change X to a consecutive number; for referencing this chapter elsewhere, use \cref{ChapterX}

This report started with the question, loosely phrased, "Is it really necessary to have as many simulation tools as there currently are in the marketplace?".
Having reviewed three rather different simulation tools, the answer seems to be a qualified "maybe".

As we have discussed, the primary difference between software tools for simulation lies in how the model is specified, and the ease of configuring and visualizing the simulation.
Certainly, the argument can be made that this process is different for different fields of science and engineering, and that this justifies having highly specialized tools.
Even so, the author would argue that a general modelling language, such as Modelica, could readily be used in specialized software.
We reviewed one such specialized tool, OpenCOR, which is intended for biological sciences.
In the opinion of the author, this project might well have been better served by using an existing modelling language, freeing up development time to be used for other parts of the project.
We have also seen a standard interface for model sharing, which has gained a lot of traction in the industry, namely FMI, and how we can simulate FMUs from an external program.

\end{document}
