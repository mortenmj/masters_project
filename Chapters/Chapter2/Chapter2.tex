% Chapter Template
\providecommand{\rootfolder}{../..} % Relative path to main.tex
\documentclass[\rootfolder/main.tex]{subfiles}
\begin{document}
\chapter{Historical background} % Main chapter title

\label{ch:background} % Change X to a consecutive number; for referencing this chapter elsewhere, use \cref{ChapterX}

\section{Human Computers}

\begin{wrapfigure}{R}{0.5\columnwidth}
    \pimage[0.48]{Figures/tabulatingroom}
    \caption[A computing room in the 1920s.]
            {A computing room in the 1920s. \\ Courtesy of the Library of Congress\label{fig:tabulatingroom}.}
\end{wrapfigure}

In order to better understand the history of simulation, it is useful to take a look at the history of computation in general.
Analog counting devices, such as the abacus, have been used for thousands of years to speed up computations.
Originally, however, the term computer referred to people who were employed to perform calculations.
One of the earliest examples of this was the work done to predict the return of Halley's comet, by Alexis-Claude Clairaut.
Halley and Newton had correctly deduced that other celestial bodies influenced the comet's orbit.
However, solving a three-body problem, which was necessary in order to predict the date the comet would return, was considered insurmountable at the time.
Clairaut, along with two friends, used successive approximation to solve the problem \cite{wilson1993}.
The method they used is strikingly similar to the numerical methods employed by modern simulation software.

\section{Analog Computers}

Later, in 1821, Charles Babbage was employed by the Astronomical Society of London to calculate logarithmic tables.
This had been done before, in particular by the French, who had perfected the process of breaking down problems into their simplest parts, and set large groups of computers to work on them.
However, Babbage was disappointed by the number of mistakes made by his human computers.
He took the approach to its next logical step and began to work on a calculating machine; An analog computer that could calculate the values of a polynomial simply by using repeated addition.
While not particularly successful, Babbage's second machine, the Analytical Engine, included most of the major components of a modern computer, such as input/output, memory and a central processing unit.

Later, the two world wars would bring both a greater need for computation, as well as a scarcity of manpower.
This pushed the development of increasingly complex computing machines.
Computers in the early 20th century heavily relied on complex mechanical calculators and punched card machines in order to keep up with the growing demand for computation.
The punched card machine was capable of interpreting data, and could operate at full speed for days unlike a human computer that would inevitably tire \cite{carr}.

\section{Digital Computers}

\begin{wrapfigure}{L}{0.6\columnwidth}
    \pimage[0.58]{Figures/eniac}
    \caption[Technicians programming the ENIAC.]
            {Technicians programming the ENIAC. \\ Courtesy of Los Alamos National Laboratory\label{fig:eniac}.}
\end{wrapfigure}

Although punched card machines were much faster than human computers, as electro-mechanical machines they were slow compared to the next innovation in computing: the electronic general-purpose computer.
In 1943, the United States Army commissioned engineers at the University of Pennsylvania to build the Electronic Numerical Integrator and Computer, ENIAC \cite{sep-computing-history}\cite{reed1952}.
While it was not the first programmable computer (that honor belongs to the Colossus, built by the British code breakers at Bletchley Park\cite{winegrad1996}), the ENIAC was the first truly general-purpose computer.
Originally built to compute firing tables for artillery, the machine was soon used to solve other problems. 
In particular, by the nuclear scientists at Los Alamos National Laboratory, who had only recently developed the nuclear bombs dropped on Japan.
The very first real-world application of the machine would not be artillery tables, but simulations of nuclear fusion reactions \cite{AtomicHeritageFoundation}.

In 1948, a new control system was planned and implemented for the ENIAC.
The new system, based on the work of John von Neumann \cite{VonNeumann1993} \cite{Haigh2014a}, added an instruction set to the computer.
This greatly simplified its operation and gave birth to the first stored-program computer \cite{Rope2007}.

Computers at the time were programmed in assembly language, which is a slow and error-prone process.
Later, the creation of one of the first widely adopted programming languages, FORTRAN, short for Formula Translation, was motivated by a desire to make it easier to program computers to do calculations.
FORTRAN is considered the first high-level language to be developed, and was the first language to use a compiler.
In the late 70s, Matlab was developed to allow students to use linear algebra libraries, written in FORTRAN, without having to learn FORTRAN itself \cite{Moler}.
Matlab is one of the tools that will be reviewed in this report.

Many other languages have been developed since then, some for the specific intent of developing simulation models.
Three simulation frameworks with their corresponding input languages will be presented in this paper: Modelica, Matlab and OpenCOR.
A fundamental understanding of computers and programming languages is assumed in the following report.

\end{document}
