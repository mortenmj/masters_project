% Chapter Template
\documentclass[../main.tex]{subfiles}
\begin{document}
\chapter{Commonalities between modern simulation packages} % Main chapter title
%\chapter{Requirements for a general purpose simulator} % Main chapter title

\label{Chapter02} % Change X to a consecutive number; for referencing this chapter elsewhere, use \ref{ChapterX}

%----------------------------------------------------------------------------------------
%   SECTION 1
%----------------------------------------------------------------------------------------

\section{Main Section 1}

For any general purpose simulation suite to establish itself in the already
crowded marketplace, it will need to be highly customisable. In order to
provide all the benefits of existing simulation suites, without simply offering
a toolbox, the simulation suite needs a plugin-based architecture. Features
such as input- and output formats, the graphical interface, available solvers,
language packs and model components should all be customisable by the user, so
that the software can be tailored to different needs.

A weakness in current solutions is that they are either too generic or too
specific. Matlab is an example of the former, where users need a lot of
technical know-how in order to write simulations from scratch. The Matlab
language allows for any conceivable simulation problem to be run, but requires
that the user knows the language, and that the problem is described at a low
abstraction level. Other solutions are highly tailored to a particular field,
or even sub-field, and only allow the user to simulation a constrained set of
problems. These software suites typically allow the user to quickly simulate a
problem, as long as it is within the realm of problems supported by the
software, often with a graphical interface so that no programming experience is
required of the user.

The benefit of the former is, of course, that any problem can be simulated
given that the user has the required skill set. In practice however, the result
is frequently that a software solution is cobbled together by end users with
little experience in the best practises of software architecture. This leads to
what software architects affectionately refer to as a big ball of mud.
Conversely, the latter approach allows users to quickly simulate problems,
without much knowledge of the software, but leaves the user with little
opportunity to customise the application if needed.

In order to better understand how these requirements might be met by a single
piece of software, the following chapter will look at some industry leading
solutions for inspiration.

%-----------------------------------
%   SUBSECTION 1
%-----------------------------------
\subsection{Subsection 1}

Nunc posuere quam at lectus tristique eu ultrices augue venenatis. Vestibulum ante ipsum primis in faucibus orci luctus et ultrices posuere cubilia Curae; Aliquam erat volutpat. Vivamus sodales tortor eget quam adipiscing in vulputate ante ullamcorper. Sed eros ante, lacinia et sollicitudin et, aliquam sit amet augue. In hac habitasse platea dictumst.

%-----------------------------------
%   SUBSECTION 2
%-----------------------------------

\subsection{Subsection 2}
Morbi rutrum odio eget arcu adipiscing sodales. Aenean et purus a est pulvinar pellentesque. Cras in elit neque, quis varius elit. Phasellus fringilla, nibh eu tempus venenatis, dolor elit posuere quam, quis adipiscing urna leo nec orci. Sed nec nulla auctor odio aliquet consequat. Ut nec nulla in ante ullamcorper aliquam at sed dolor. Phasellus fermentum magna in augue gravida cursus. Cras sed pretium lorem. Pellentesque eget ornare odio. Proin accumsan, massa viverra cursus pharetra, ipsum nisi lobortis velit, a malesuada dolor lorem eu neque.

%----------------------------------------------------------------------------------------
%	SECTION 2
%----------------------------------------------------------------------------------------

\section{Main Section 2}

Sed ullamcorper quam eu nisl interdum at interdum enim egestas. Aliquam placerat justo sed lectus lobortis ut porta nisl porttitor. Vestibulum mi dolor, lacinia molestie gravida at, tempus vitae ligula. Donec eget quam sapien, in viverra eros. Donec pellentesque justo a massa fringilla non vestibulum metus vestibulum. Vestibulum in orci quis felis tempor lacinia. Vivamus ornare ultrices facilisis. Ut hendrerit volutpat vulputate. Morbi condimentum venenatis augue, id porta ipsum vulputate in. Curabitur luctus tempus justo. Vestibulum risus lectus, adipiscing nec condimentum quis, condimentum nec nisl. Aliquam dictum sagittis velit sed iaculis. Morbi tristique augue sit amet nulla pulvinar id facilisis ligula mollis. Nam elit libero, tincidunt ut aliquam at, molestie in quam. Aenean rhoncus vehicula hendrerit.

\end{document}
