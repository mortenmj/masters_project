% Chapter Template
\providecommand{\rootfolder}{../..} % Relative path to main.tex
\documentclass[\rootfolder/main.tex]{subfiles}
\begin{document}

\chapter{Introduction} % Main chapter title

\label{ch:introduction} % Change X to a consecutive number; for referencing this chapter elsewhere, use \ref{ChapterX}

The are many simulation languages, and software tools for constructing and simulating models in these languages, available to engineers and scientists.
In this report, this situation is investigated by comparison of three selected software tools, with the aim of answering if this is necessary.

The report will first take a brief look at the history of the computer and of programming languages, as it relates to simulation.
Here, it will be shown how the need to simulate physical systems has been a driving force in the development of the computer, and an attempt will be made to cast some light on why there are so many software solutions in the field.
Then, three software solutions will be compared, in terms of their similarities and differences as well as their strengths and weaknesses.
\Cref{ch:modelica} will review the Modelica language.
In \cref{ch:matlab} the Matlab computing environment and the included Simulink environment is reviewed.
Finally, \cref{ch:opencor} will review OpenCOR, including the CellML and SED-ML languages.
An attempt will be made to craft some suggestions as to what features from these should be found in a software tool that aims to support the widest possible range of engineering disciplines.

\end{document}
