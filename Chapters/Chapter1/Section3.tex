%----------------------------------------------------------------------------------------
%   SECTION 3
%----------------------------------------------------------------------------------------

\documentclass[../../main.tex]{subfiles}
\begin{document}
\section{Digital computers}

\begin{figure}[ht]
    \cimage{eniac}{Courtesy of Los Alamos National Laboratory}
    \caption{Technicians programming the ENIAC.}
    \label{fig:eniac}
\end{figure}

While punched card machines were much faster than human computers, as electro-mechanical machines they were slow compared to the next innovation in computing: the electronic general-purpose computer.
In 1943, the Ordnance Corps, Research and Development Command of the United States Army commissioned engineers at the University of Pennsylvania's Moore School of Electrical Engineering
to build the Electronic Numerical Integrator and Computer, ENIAC \cite{sep-computing-history}\cite{reed1952}.
ENIAC, designed by John Mauchly and J. Presper Eckert, would not be the first programmable computer, as that honor belongs to the Colossus,
built by the British codebreakers at Bletchley Park\cite{winegrad1996}, but it would be the first truly general-purpose computer.
In fact, though originally designed to compute firing tables for the army, the the computer was general enough in its design that it could calculate a wide variety of numerical problems \cite{10.2307/2002620}
The machine was soon put to use solving other problems, in particular by the nuclear scientists at Los Alamos National Laboratory, who had only recently developed the nuclear bombs dropped on Japan.
The very first real-world application of the machine would not be computing artillery tables, but by Edward Teller at Los Alamos who used the machine to simulate nuclear fusion reactions \cite{AtomicHeritageFoundation}.

The ideas developed by the people who had worked on the ENIAC, and its successor the EDVAC, were disseminated in 1946 at the Moore School Lectures.
Scientists at the Moore School had already begun work on the EDVAC, which would be the first binary computer,
and the school was central to many of the earliest developments in digital computing.
The attendees at the lectures were all established academics, physicists and mathematicians and the lectures led to an explosion of activity in both the United States and Europe \cite{Davis2008}.

The research in nuclear physics started during the war continued in peacetime and scientists at the Los Alamos National Laboratory had a large backlog of computing needs.
There was a desire to apply statistical methods to solve some of these problems.
Among the problems in the laboratory's backlog were a number of problems that could not feasibly be solved by the standard methods of the time, including the neutron permeability of various materials.
Many such problems can be solved using a statistical technique called the Monte Carlo method, but the method is computationally intensive and requires a large number of human computers.
Now that laboratory scientists had access to an electronic computer, there was a desire to use the ENIAC to do the calculations \cite{Haigh2014}.
However, the ENIAC was designed as a large collection of functional modules, and was programmed by manually connecting these to one another with a large switchboard.
This meant that while the computer was very fast by the standards at the time, setting the machine up to solve a new problem could take weeks \cite{Rope2007}.
Clearly, this would not suffice if scientists were to apply the machine to a wide range of problems.

In 1948, a new control system was planned and implemented for the ENIAC.
The new system, based on the work of John von Neumann \cite{VonNeumann1993} \cite{Haigh2014a} added an instruction set to the computer.
This greatly simplified its operation and arguably gave birth to the first stored-program computer \cite{Rope2007}.
In this way, the need to quickly simulate a variety of mathematical problems directly led to the creation of the modern general-purpose computer.

\end{document}
