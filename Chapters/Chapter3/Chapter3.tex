% Chapter Template
\providecommand{\rootfolder}{../..} % Relative path to main.tex
\documentclass[\rootfolder/main.tex]{subfiles}
\begin{document}
\chapter{Matlab/Simulink} % Main chapter title

\label{Chapter03} % Change X to a consecutive number; for referencing this chapter elsewhere, use \cref{ChapterX}

\section{Language features}

While Matlab offers many different programming paradigms, and both numerical and symbolic mathematics, it is most known as a numerical scripting language.
For this reason, we will first and foremost review these capabilities in the language.

In addition to the Matlab programming language, Matlab also allows for modelling using graphical blocks in Simulink.
However, it is noteworthy that these are not one and the same.
While we saw in \cref{Chapter02} that in Modelica, graphical models were a representation of a text based specification, in Matlab these are separate things.
In Matlab, these are separate.
We may create models using state space representations, transfer functions, algorithms or a combination thereof in the Matlab language.
We may also create models using graphical components, either causal or acausal as we shall see.
Furthermore, models created in the Matlab language can be included as components in a graphical model, using what is called an s-function.
However, a Simulink model is in a binary format, and there is no underlying Matlab code that may be edited.

To illustrate some features of the language, we will revisit the example used previously, with the coupled driveshaft.
The equations for this system were shown in \cref{eq:inertial}.
To implement this system in Matlab, we first change it into matrix form.
If we let $\vec{x}{t}$ be the states, $\vec{u}{t}$ be the inputs and $\vec{y}{t}$ be the outputs of the system, any such system can be represented in the following form:

\loadeq{\rootfolder/Equations.tex}{003} % State space representation of the system in vector/matrix form

Matlab allows for systems to be modelled in several different ways.
To illustrate this, we will return to the same drive shaft example as used in \cref{Chapter02}.
The equations shown in \cref{eq:inertial} can be transformed into state-space form.

\Cref{lst:inertialmatlab} shows \cref{eq:inertial} implemented in the Matlab language, with the same example values used in \cref{fig:inertial};

While perhaps most known as a scripting language, Matlab offers many features from object-oriented languages.

\begin{listing}[ht]
    \inputminted[fontsize=\footnotesize, firstline=1, lastline=9]{matlab}{\rootfolder/Models/Matlab/InertialMatlab.m}
    \caption{Inertial system implemented numerically in Matlab\label{lst:inertialmatlab}}
\end{listing}

This creates a state space object which can be used with a number of other methods available in Matlab.
It is also possible to transform this into a transfer function object, and use any number of Matlab methods made for analysing these.
The system can be analyzed to find initial response, step response and so forth.
Additionally, Matlab optionally includes toolboxes with a rich set of tools for control system creation, which can analyze the system and help the system designer do controller tuning.

\section{Simulink modelling}

In addition to specifying the state space or transfer function representations of the system in the Matlab language, it is possible to specify
the system equations using block components in the graphical interface Simulink.
The components used in Simulink are purely causal, and as such the system must be built from components such as adders, multipliers and integrators.
To compare this with the acausal modelling paradigm shown earlier, the same system is also implemented in Simulink.

Generally, the simplest approach when implementing a system of differential equations in Simulink is to start with the integration blocks, and work backwards.
Unfortunately, as the model must be made from low-level components the possibility of making a mistake if that much greater.
The author needed significantly more time to create this model compared to the implementation in \ref{Chapter02}.

\begin{figure}[ht]
    \centering
    \tiny
    \includesvg[width=0.8\columnwidth]{Figures/InertialSimulink}
    \caption{Inertial system implemented in Simulink.\label{fig:inertial-simulink}}
\end{figure}

The block diagram equivalent of \cref{eq:inertial} is shown in \cref{fig:inertial-simulink}.
Note that the input torque is added to the term for $J_{1} \der{\omega_{1}}$.
Whereas adding an input term in Modelica required nothing more than to connect the input source to the left-hand side of the first inertia component,
adding an input term in this model requires understanding the underlying equations and correctly identifying the proper way to change the model.

We expand the model in the same was as we did previously, by adding a reference source and a controller to drive the output state of the system to the desired value.
The plant has been turned into a subsystem, to reduce clutter in the model.
Note that the plant outputs $\omega$ directly, while previously we added a sensor which differentiated $\varphi$ to solve for $\omega$.
As Matlab solves this numerically, adding a differentiator is troublesome and was skipped in the example.

\begin{figure}[ht]
    \centering
    \footnotesize
    \includesvg[width=0.8\columnwidth]{Figures/InertialSimulinkExpanded}
    \caption{Inertial system expanded.\label{fig:inertial-simulink-expanded}}
\end{figure}

In \cref{Chapter02} we saw how partial models could be instantiated with specific components.
This is useful for example in order to compare variants of two controllers.
In order to replace parts of models in Simulink there are a couple of approaches that can be used.
First, for simple components, we may use the \emph{replace\_block()} function.
\Cref{lst:matlab-replace} shows how to use this to replace the step function in \cref{fig:inertial-simulink-expanded}, to swap in a different source.

\begin{listing}[ht]
    \inputminted[fontsize=\footnotesize, firstline=1, lastline=4]{matlab}{\rootfolder/Models/Matlab/ReplaceSource.m}
    \caption{Replacing the reference signal source programmatically.\label{lst:matlab-replace}}
\end{listing}

This works for simple changes, but is limited to single blocks.
To replace subcomponents, it is necessary to introduce a variant subsystem.
This is a special type of subsystem that contains one or more plug-compatible variants.
Simulink connects one of the variants the the inputs and outputs of the subsystem based on the value of a variable.
This allows for programmatic switching between the variants.
The selection of a variant can also be made conditional on the choice of another component variant, or any other logical expression for that matter.
For example, we may configure the system to use a Kalman filter only when a certain controller variant is active.
Note that the example also shows an empty controller variant, which allows us to run the system with no controller active by choosing this variant.

\begin{figure}[ht]
    \centering
    \subcaptionbox{Model overview.\label{fig:inertial-simulink-variant-top}}
    {
        \footnotesize
        \includesvg[width=0.8\columnwidth]{Figures/InertialSimulinkVariant1}
    }
    \subcaptionbox{View of the controller subsystem.\label{fig:inertial-simulink-variant-bottom}}
    {
        \footnotesize
        \includesvg[width=0.8\columnwidth]{Figures/InertialSimulinkVariant2}
    }
    \caption{Inertial model with variant controller subsystem.\label{fig:inertial-simulink-variant}}
\end{figure}

This can be combined with running simulations of Simulink models from Matlab scripts in order to compare the effect of various signal sources on the behavior of the model.

While, as we have seen, the model can be made modular and components can have multiple implementations, but there is no mechanism in Matlab/Simulink for
the types of partial model implementation shown in \cref{fig:inertial-architecture}.
In effect, this means that while we can create highly configurable models, any such changes requires that the original model is edited.

Furthermore, the Simulink file format is a binary, as opposed to a text, format.
Therefore, it is not possible to do these changes with a simple text replacement, like we could have done with the model from \cref{lst:inertial-instantiation}.

\section{Simscape modelling}

In addition to the causal modelling done in Simulink above, Matlab also comes with an expansion called Simscape.
In Simscape, we can model using acausal components like those we saw in \cref{Chapter02}.
\Cref{fig:inertial-simscape} shows what our model looks like in Simscape.

\begin{figure}[ht]
    \iimage{Figures/InertialSimscape}
    \caption{Inertial system in Simscape.\label{fig:inertial-simscape}}
\end{figure}

Simscape components are not signal-compatible with those in Simulink, but using special converter components we can interface Simscape models with Simulink models.
This way, we might for example connect the acausal plant implementation above to a causal controller implementation.
As acausal modelling has already been covered in \cref{Chapter02}, Simscape modelling will not be covered in further detail.

\section{Simulation}

When simulating a model in Matlab the procedure is somewhat different depending on whether the model is defined in Matlab code, or in Simulink or Simscape.

\subsection{Simulink}

\subsubsection{Simulink model compilation}

When simulating a simulink model, the compiler first determines the values of any variables used in the model.
In our example, several values are set in a Matlab script and the compiler must populate the model with the numerical values before proceeding.
In the case of symbolic expressions, these must be resolved to determine the resulting numerical value that is used in the simulation.
The data type, numeric type and dimensions of signals are also checked, to ensure that every block in the model is compatible with the signals that are applied to it.

\begin{figure}[ht]
    \centering
    \tiny
    \includesvg[width=0.8\columnwidth]{Figures/InertialSimulinkOrder}
    \caption{Execution order for the plant subsystem.\label{fig:inertial-simulink-order}}
\end{figure}

After this, the algorithm determines the sorted execution order for the model, so that blocks are executed in an order which reflects their data dependencies.
This turns the graphical model into a sequentially executed program that can be simulated by a computer.
\Cref{fig:inertial-simulink-order} shows the execution order for the model we have been working with.
The execution order is determined by the algorithm, but the designer can exert some influence on it by assigning higher priority a block.
It is also possible to designate a subsystem as atomic, so that the whole subsystem is executed as a single unit.
Beyond this virtual blocks, such as subsystems, are flattened and do not influence the execution order.

Finally, the sample times are determined for blocks that have inherited sample times.

\subsubsection{Model linking}

After the model is compiled, the Simulink engine allocates working memory for the model execution, as well as memory for run-time data that is to be stored.
The method execution list, which lists the most efficient execution order of the model blocks, is generated from the sorted order lists found previously.

\subsubsection{Loop phase}

After model linking, the algorithm proceeds to the simulation loop phase.
The loop phase is separated into two subphases, the initialization phase and the iteration phase.

First, during initialization, the system determines the initial states for the system and the output values that need to be computed.
Then, during the loop iteration phase, the following takes place:

\begin{enumerate}
    \item Computation of outputs
    \item Computation of the model states
        This step is performed by invoking a solver.
        Which solver depends on whether the model has no states, only continuous states, only discrete states or both types of states.
        The user can specify which continuous and discrete solver to use, or the system can be asked to determine the optimal solver automatically.
    \item Discontinuity checks
    \item Computation of the time until the next time step
\end{enumerate}

These steps are repeated until the simulation end-time.

\subsection{Simscape}

Simulation of Simscape models is somewhat more involved, as the model is acausal.

\subsubsection{Model validation}

Some model validation is performed above, but for acausal models the procedure is more involved.
When simulating a Simscape model, the system ensures that all blocks are connected.
It is not allowed to leave conserving ports, such as those representing hydraulic flow or energy flow, unconnected.
Models containing fluid or gas flow networks require that no more than one block which specifies the properties of the fluid or gas is connected to each network.

\subsubsection{Network construction}

After this, the system creates the physical network.
This is based on two types of properties.

\begin{itemize}
    \item Through-properties (e.g. current) are those which are measured in series with an element
    \item Across-properties (e.g. voltage) are those which are measured in parallel with an element
\end{itemize}

The system follows the rules that for every through-variable, the sum of the values going into a node equal those going out.
Conversely, for every across-variable, two directly connected elements have the same value.
The system creates the physical network following these two principles.

\subsubsection{Equation construction}

The system equations are then developed from this network, and the equations are simulated in a similar fashion to the procedure explained for normal Simulink models.
Overall, the process is similar to that described in \cref{Chapter02}.

\end{document}
