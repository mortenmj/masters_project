%-----------------------------------
%   Input Formats
%-----------------------------------

\providecommand{\rootfolder}{../..} % Relative path to main.tex
\documentclass[\rootfolder/main.tex]{subfiles}
\begin{document}
\section{Input formats}

The two most powerful methods for describing systems, and also the most common, are the state space representation and the transfer function representation.
While classical control theory is based on transfer function analysis, it has several drawbacks.
Modern control theory largely relies on state space representation, which benefits from several powerful techniques in linear algebra.
While both methods can be used to represent linear time-invariant systems, the state space representation offers several attractive advantages:

While it is possible to extend the notion of transfer functions to time-variant and non-linear systems, it is rarely done.
Representing a nonlinear system requires a sequence of transfer functions \cite{Zhang1993}, and the stronger the nonlinearities are the more functions are required.
For this reason, the use of transfer functions is largely restricted to linear, time-invariant systems.

\begin{itemize}
	\item To treat multivariate systems in transfer function form requires a transfer function matrix. This can become unwieldy.
		Using the state space representation allows us to treat univariate and multivariate systems the same way.
	\item For any transfer function, there are many possible state space representations. Therefore, converting to the transfer function form involves a loss of information.
		The state space representation maintains this information, allowing greater insight into the inner workings of the system.
	\item The state space representation allows for non-zero initial conditions, unlike a transfer function representation.
	\item The state space representation is defined for linear and non-linear systems, unlike a transfer function representation which is only defined for linear systems.
	\item The state space representation is defined for time-variant and time-invariant systems, unlike a transfer function representation which is only defined for time invariant systems.
	\item Given the state space representation, it is simple to determine if the system is observable and controllable.
\end{itemize}

To see how a state space representation may be represented in software, it is useful to review how it is defined.
A system with states $x_{1}$ to $x_{p}$ and outputs $y_{1}$ to $y_{r}$ are related by the following equations:

\loadeq{\rootfolder/Equations.tex}{001}

We can rearrange this to represent the variables as vectors, and the coefficient as matrices:
\loadeq{\rootfolder/Equations.tex}{002}

Finally, in compact form, this may be written as:
\loadeq{\rootfolder/Equations.tex}{003}

As we can see, this system can be uniquely identified in the software by the matrices $(A, B, C, D)$, and it follows that this would be all we need to pass to a solver in order to simulate the system.
However, in order to separate the model from the simulation setup, and to promote sharing of models, a richer input format would be desirable.
In order to promote sharing of models, it is useful to attach metadata to the model, such as units.
It would also be desirable to store the model in a format where variables and coefficients can have descriptive names.
While the format described above, where the model is defined by its coefficient matrices, is sufficient to define the model mathematically, it is insufficient in these regards.
In order to fulfill all of these requirements, storing the model as MathML would be more feasible.
An input method like the one employed by OpenCOR, which we reviewed earlier, would be one possible solution, as shown in %\cref{fig:opencor-input}.

\end{document}
