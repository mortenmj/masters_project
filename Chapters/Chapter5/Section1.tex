%----------------------------------------------------------------------------------------
%   General Requirements
%----------------------------------------------------------------------------------------

\providecommand{\rootfolder}{../..} % Relative path to main.tex
\documentclass[\rootfolder/main.tex]{subfiles}
\begin{document}
\section{General Requirements}

For any general purpose simulation suite to establish itself in the already crowded marketplace, it will need to be highly customisable.
In order to provide all the benefits of existing simulation suites, without simply offering a toolbox, the simulation suite needs a plugin-based architecture.
Features such as input- and output formats, the graphical interface, available solvers, language packs and model components should all be customisable by the user,
so that the software can be tailored to different needs.

A weakness in current solutions is that they are either too generic or too specific.
Matlab is an example of the former, where users need a lot of technical know-how in order to write simulations from scratch.
The Matlab language allows for any conceivable simulation problem to be run, but requires that the user knows the language,
and that the problem is described at a low abstraction level.
Other solutions are highly tailored to a particular field, or even sub-field, and only allow the user to simulation a constrained set of problems.
These software suites typically allow the user to quickly simulate a problem, as long as it is within the realm of problems supported by the software,
often with a graphical interface so that no programming experience is required of the user.

The benefit of the former is, of course, that any problem can be simulated given that the user has the required skill set.
In practice however, the result is frequently that a software solution is cobbled together by end users with little experience in the best practises of software architecture.
This leads to what software architects affectionately refer to as a big ball of mud.
Conversely, the latter approach allows users to quickly simulate problems, without much knowledge of the software, but leaves the user with little opportunity to customise the application if needed.

In order to better understand how these requirements might be met by a single piece of software, we will look at each of them in turn.

\end{document}
